\documentclass[12pt, a4paper]{article}

\usepackage{CJKutf8}
\title{Timelog}
\author{105820009 汪秉沄\\
108598074 顏翊純\\
105820014 張雅雯\\
105820026 張景博\\
108598047 胡喻翔}
\date{\today}

\begin{document}
\begin{CJK*}{UTF8}{bkai}
\pagenumbering{gobble}
\maketitle
\newpage

\tableofcontents
\newpage

\pagenumbering{arabic}

\section{Requirement Document}
  \subsection{Change History}

  \subsection{Problem Statement}

  \subsection{System Context Diagram}

  \subsection{Summary of System Feature}

  \subsection{Use Case Diagram}

  \subsection{Use Cases}
    \begin{itemize}
      \item {{Template item}}
        \begin{itemize}
          \item Scope:
          \item Level:
          \item Primary Actor:
          \item Stakeholders and Interests:
          \item Preconditions:
          \item Success Guarantee:
          \item Main Success Scenario:
          \item Extensions:
          \item Special Requirements:
          \item Technology and Data Variations List:
          \item Frequency of Occurrence:
          \item Miscellaneous:
        \end{itemize}
      \item Manage user account.
        \begin{itemize}
          \item Scope: Account Management System
          \item Level: subfunction
          \item Primary Actor: User
          \item Stakeholders and Interests:
            \begin{itemize}
              \item {\bf Member}: A teammeber wants to login to Timelog to record and maintain their own logs.
              \item {\bf Manager}: A manager wants to be able to maintain the members of team, and see teammebers' timelog report.
              \item {\bf Organization}: With several teams in an organization, we a group to association with user account to maintain the members easily.
            \end{itemize}
          \item Preconditions: None.
          \item Success Guarantee:
            \begin{itemize}
              \item User register account
              \item User login to system.
              \item Logs are distinguished by users identity.
              \item Manager maintain users of the team.
              \item Users see the logs reports of other team members of the same team.
            \end{itemize}
          \item Main Success Scenario: User login to timelog
            \begin{enumerate}
              \item User enter Timelog page.
              \item Direct user to login page.
              \item User select a login method to login to system.
                \begin{itemize}
                  \item {\bf FB login}
                    \begin{enumerate}
                      \item System redirect user to FB login page.
                      \item User enter Facebook account info to login.
                    \end{enumerate}
                  \item {\bf Google login}
                    \begin{enumerate}
                      \item System redirect user to Google login page.
                      \item User enter Google account info to login.
                    \end{enumerate}
                  \item {\bf Github login}
                    \begin{enumerate}
                      \item System redirect user to Github login service.
                      \item User enter Github account info to login.
                    \end{enumerate}
                \end{itemize}
              \item System redirects the user back to Timelog page.
              \item User logined to timelog.
            \end{enumerate}
          \item Extensions:
            \begin{enumerate}
              \item User register an account
              \begin{enumerate}
                \item User enters to Timelog page.
                \item System redirects user to login page.
                \item User selects to register an account.
                \item User inputs the basic information.
                \item User creates an account
              \end{enumerate}
              \item User autologin
                \begin{enumerate}
                  \item User enter to Timelog page.
                  \item If user login with same device before.
                  \item The System auto login for user.
                  \item User enter the system.
                \end{enumerate}
              \item If user give the wrong login information.
                \begin{enumerate}
                  \item User enter to Timelog page.
                  \item User entered incorrect account or password
                  \item System reject login.
                \end{enumerate}
              \item Manager manage team members of a team.
                \begin{enumerate}
                  \item Manager enter team manage page.
                  \item Manager do manage operations to manage the members in the team.
                    \begin{itemize}
                      \item Add user to the team.
                      \item Remove a member from the team.
                      \item Set role of the team to a member.
                    \end{itemize}
                  \item The team members' team information updated.
                \end{enumerate}
            \end{enumerate}
          \item Special Requirements:
            \begin{enumerate}
              \item Provide an identity to distinguish logs.
              \item User should be able to connect their account with several third part oauth services, such as FB, Google, Github.
              \item Record user's login status.
            \end{enumerate}
          \item Technology and Data Variations List:
            \begin{enumerate}
              \item User can login through thee kinds of third part login, including Facebook, Google, and Github.
              \item A user may have role in a team, manager or normal member, with different authority to team management
            \end{enumerate}
          \item Frequency of Occurrence: Every time an user wants to use Timelog.
          \item Miscellaneous:
            \begin{itemize}
              \item Open Issues:
                \begin{itemize}
                  \item How long should user token expired?
                  \item What roles should we have for members in a team?
                \end{itemize}
            \end{itemize}
        \end{itemize}
      \item User manage logs.
        \begin{itemize}
          \item Scope: Timelog system
          \item Level: User-goal
          \item Primary Actor: user
          \item Stakeholders and Interests:
            \begin{itemize}
              \item {\bf User}: User wants to record his/her life activities to monitor him/her self time usage to improve his/her time management.
            \end{itemize}
          \item Preconditions:
            \begin{itemize}
              \item User logined
              \item User set activity types.
            \end{itemize}
          \item Success Guarantee:
            \begin{itemize}
              \item User record a log to the system.
              \item User edit a log in the system.
              \item User delete a log in the system.
            \end{itemize}
          \item Main Success Scenario: User record a log.
            \begin{enumerate}
              \item User logined to Timelog.
              \item User send "Record Log" command
              \item User input informations needed for a log.
                \begin{itemize}
                  \item Title
                  \item Time period
                  \item Activity Type
                  \item Description
                \end{itemize}
            \end{enumerate}
          \item Extensions:
            \begin{enumerate}
              \item User edit log.
                \begin{enumerate}
                  \item User select an incorrect log.
                  \item User inputs the new information of log.
                  \item User update the selected log.
                \end{enumerate}
              \item User delete a log
                \begin{enumerate}
                  \item User select a log.
                  \item User delete the selected log.
                \end{enumerate}
            \end{enumerate}
          \item Special Requirements: None
          \item Technology and Data Variations List:
            \begin{itemize}
              \item Activity types are several types defined by the user.
            \end{itemize}
          \item Frequency of Occurrence: an hour to a day.
          \item Miscellaneous: None
        \end{itemize}
      \item System generates report.
        \begin{itemize}
          \item Scope: Timelog System
          \item Level: User-goal
          \item Primary Actor: Timelog System
          \item Stakeholders and Interests:
            \begin{itemize}
              \item {\bf User}: User see the report of selected period or timebox, so he/she can know how did his/her time have been spent.
              \item {\bf Manager}: Manager wants to see the the time allocation of his/her team members, to analysis the working quality of the team.
            \end{itemize}
          \item Preconditions:
            \begin{itemize}
              \item Log added.
              \item Activity types added.
              \item Timebox created.
              \item Goal of activities set.
            \end{itemize}
          \item Success Guarantee:
            \begin{itemize}
              \item Pie chart and table report generated.
              \item the data of report is in the range of selected period/Timebox
              \item Goal complete rate report generated.
            \end{itemize}
          \item Main Success Scenario:
            \begin{enumerate}
              \item User's operations satisfied all preconditions.
              \item System generates the report according to Timebox, activity types, goal, and logs.
              \item Reports showed on web page.
            \end{enumerate}
          \item Extensions:
            \begin{enumerate}
              \item Report without goal.
                \begin{enumerate}
                  \item If user didn't set goal for current Timebox.
                  \item System generates the report accoring to Timebox, activity types, and logs(wihtout goal).
                  \item Reports showed on web page (wihtout goal).
                \end{enumerate}
            \end{enumerate}
          \item Special Requirements:
            \begin{enumerate}
              \item User can specify a period, instead of Timebox to generate report.
            \end{enumerate}
          \item Technology and Data Variations List:
            \begin{itemize}
              \item Duration of report could be Timebox, or user selected period.
              \item Goal report showed only if user sets at least one goal.
            \end{itemize}
          \item Frequency of Occurrence: Every time user add a new log.
          \item Miscellaneous:
            \begin{itemize}
              \item Should we provide more than one chart rather than pie chart only?
              \item What information to show on report?
            \end{itemize}
        \end{itemize}
      \item User manage Activity Type.
        \begin{itemize}
          \item Scope: Timelog System
          \item Level: User-goal
          \item Primary Actor: user
          \item Stakeholders and Interests:
            \begin{itemize}
              \item {\bf User}: User needs activity type
            \end{itemize}
          \item Preconditions: User logined to Timelog.
          \item Success Guarantee:
            \begin{itemize}
              \item User add an Activity Type
              \item User edit an Activity Type
              \item User delete an Activity Type
            \end{itemize}
          \item Main Success Scenario: User add an activity type.
            \begin{enumerate}
              \item User logins to Timelog
              \item User enters Activity Type management page
              \item User inputs information for new Activity Type.
              \item User triggers "add" command.
              \item The new Activity Type added to user's account.
            \end{enumerate}
          \item Extensions:
            \begin{itemize}
              \item At any time, if user leaves setting page, the add operation is canceled.
              \item User edits an Activity Type.
                \begin{enumerate}
                  \item User logins to timelog.
                  \item User enters Activity Type management page.
                  \item User selects an existed Activity Type.
                  \item User gives an edit command.
                  \item User updates new informations for the type.
                  \item User gives an update command.
                  \item The selected Activity Type is updated.
                \end{enumerate}
              \item User deletes an Activity Type.
                \begin{enumerate}
                  \item User logins to timelog.
                  \item User enters Activity Type management page.
                  \item User selects an existed Activity Type.
                  \item User gives an delete command.
                  \item The selected Activity Type is deleted.
                \end{enumerate}
            \end{itemize}
          \item Special Requirements:
            \begin{itemize}
              \item The users should be able to open Activity Type management while adding a new log, to simplify the add log process.
            \end{itemize}
          \item Technology and Data Variations List:
            \begin{itemize}
              \item Activity Type management can either be a page, user selects the management subpage, or a popup window, user triggers while adding a log.
            \end{itemize}
          \item Frequency of Occurrence: Mostly few month or even years, according to the life event of the user.
          \item Miscellaneous:
            \begin{itemize}
              \item What information should we have except for type name?
              \item Is there a limit for number of activities?
            \end{itemize}
        \end{itemize}
      \item Export report to FTP server.
        \begin{itemize}
          \item Scope: Timelog System
          \item Level: subfunction
          \item Primary Actor: User
          \item Stakeholders and Interests:
            \begin{itemize}
              \item {\bf User}: User export and upload the report to the shared FTP server, so other members of the organization can see the log.
              \item {\bf Manager}: An organization manager can see the reports of members on a manual service based on an FTP server.
              \item {\bf Organization}: We can have a self hosted FTP server to record all the reports of organization members.
            \end{itemize}
          \item Preconditions:
            \begin{itemize}
              \item User set acitivty types
              \item User add logs
              \item User set Timebox
              \item User set goal(Not required)
            \end{itemize}
          \item Success Guarantee: The report is generated an uploaded to FTP server.
          \item Main Success Scenario: Report exported and uploaded to FTP server.
            \begin{itemize}
              \item User logins to Timelog
              \item User selects an Timebox
              \item User gives export command
              \item User sets FTP server authorization info.
              \item The system generates an report and uploads to the specified FTP server.
            \end{itemize}
          \item Extensions: None
          \item Special Requirements:
            \begin{itemize}
              \item If user exported a report before, the system should auto fill the FTP info for the user.
            \end{itemize}
          \item Technology and Data Variations List:
            \begin{itemize}
              \item Upload can through either FTP or FTPs protocol.
            \end{itemize}
          \item Frequency of Occurrence: Weekly or monthly, according to the organization's policy.
          \item Miscellaneous: None
        \end{itemize}
      \item User manage profile.
      \item User "CRUD" timebox.
      \item User "CURD" goal of timebox.
      \item Team member see the report of whole team \& other members.
      \item Manager add/delete member to group.
      \item manger "CRUD" sub team.
      \item Admin set role to users.
      \item Manager set goal for team.
      \item User set background color.
    \end{itemize}

  \subsection{Non-functional Requirements and Constraints}

  \subsection{Glossary}

  \subsection{Software Environments}

\section{The Previous Project Information}
  \subsection{Use case diagram}

  \subsection{Conceptual model}

  \subsection{Design Class diagram}

  \subsection{System features that have been developed before this project began.}
\end{CJK*}
\end{document}
